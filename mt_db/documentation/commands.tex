%%%%%%%%%%%%%%%%%%%%%%%%%%%%%%%%%%%%%%%%%
% Short Sectioned Assignment
% LaTeX Template
% Version 1.0 (5/5/12)
%
% This template has been downloaded from:
% http://www.LaTeXTemplates.com
%
% Original author:
% Frits Wenneker (http://www.howtotex.com)
%
% License:
% CC BY-NC-SA 3.0 (http://creativecommons.org/licenses/by-nc-sa/3.0/)
%
%%%%%%%%%%%%%%%%%%%%%%%%%%%%%%%%%%%%%%%%%

%----------------------------------------------------------------------------------------
%	PACKAGES AND OTHER DOCUMENT CONFIGURATIONS
%----------------------------------------------------------------------------------------

\documentclass[paper=a4, fontsize=11pt]{scrartcl} % A4 paper and 11pt font size

\usepackage[T1]{fontenc} % Use 8-bit encoding that has 256 glyphs
\usepackage{fourier} % Use the Adobe Utopia font for the document - comment this line to return to the LaTeX default
\usepackage[english]{babel} % English language/hyphenation
\usepackage{amsmath,amsfonts,amsthm} % Math packages

\usepackage{lipsum} % Used for inserting dummy 'Lorem ipsum' text into the template

\usepackage{sectsty} % Allows customizing section commands
\allsectionsfont{\centering \normalfont} % Make all sections centered, the default font and small caps

\usepackage{fancyhdr} % Custom headers and footers
\pagestyle{fancyplain} % Makes all pages in the document conform to the custom headers and footers
\fancyhead{} % No page header - if you want one, create it in the same way as the footers below
\fancyfoot[L]{} % Empty left footer
\fancyfoot[C]{} % Empty center footer
\fancyfoot[R]{\thepage} % Page numbering for right footer
\renewcommand{\headrulewidth}{0pt} % Remove header underlines
\renewcommand{\footrulewidth}{0pt} % Remove footer underlines
\setlength{\headheight}{13.6pt} % Customize the height of the header

\numberwithin{equation}{section} % Number equations within sections (i.e. 1.1, 1.2, 2.1, 2.2 instead of 1, 2, 3, 4)
\numberwithin{figure}{section} % Number figures within sections (i.e. 1.1, 1.2, 2.1, 2.2 instead of 1, 2, 3, 4)
\numberwithin{table}{section} % Number tables within sections (i.e. 1.1, 1.2, 2.1, 2.2 instead of 1, 2, 3, 4)

\setlength\parindent{0pt} % Removes all indentation from paragraphs - comment this line for an assignment with lots of text

%----------------------------------------------------------------------------------------
%	TITLE SECTION
%----------------------------------------------------------------------------------------

\newcommand{\horrule}[1]{\rule{\linewidth}{#1}} % Create horizontal rule command with 1 argument of height

\title{	
\normalfont \normalsize 
\textsc{University of Southern California} \\ [25pt] % Your university, school and/or department name(s)
\horrule{0.5pt} \\[0.4cm] % Thin top horizontal rule
\huge CS402 Assignment 01 -- Commands \\ % The assignment title
\horrule{2pt} \\[0.5cm] % Thick bottom horizontal rule
}

\author{Michael Gerow} % Your name

\date{\normalsize\today} % Today's date or a custom date

\begin{document}

\maketitle % Print the title

%----------------------------------------------------------------------------------------
%	PROBLEM 1
%----------------------------------------------------------------------------------------

\section{e Command}

The e command works by calling a function called create\_client() which calls the client\_create function in order to allocate the client.  After the client is created we then call add\_client\_to\_thread\_handler which adds the client to the thread handler's list of running clients.  In order to do this it gets a lock on the list, adds itself to it, and then unlocks.  After that we simpl call the launch\_client\_thread function which calls pthread\_create on the thread, running the client\_run code.

When the thread has finished it simply sets its (newly added) done flag to true and increments the thread handler's semaphore. This causes the thread handler to go through its list of clients looking for one to clean up. When it finds one it calls pthread\_join on the thread and cleans up any of its other resources.

\section{E Command}

The E command works very similarly to the e command with the exception that it pulls additional arguments from the command command interface in order to acquire the input and output files it needs. Once it has these it simply calls create\_non\_interactive\_client which calls client\_create\_no\_window which we, like with the e command, simply add to the thread handler's list of running clients and then launch the client in much the same way.

\section{s Command}

The s command by using a pause flag in the thread handler's data. When the command is called we simply acquire the mutex for that flag, set the flag to true, and then unlock the mutex.

In the client threads we call a function called wait\_on\_pause which first checks to see if we are paused, and if we are, acquires the pause mutex and checks again.  If we are still paused it waits on the pause\_cond condition variable which is broadcasted when the g command is run.

The intention of this double checking is to prevent unnecessary locking and unlocking of the pause mutex, which I assumed would cause performance issues.  In my tests, though, I have found only a moderate if any increase in performance due to using this technique.

\section{g Command}

The g command simply grabs the pause mutex, sets pause to false, and then broadcasts the pause\_mutex, freeing all the threads blocking on this condition and allowing them to continue execution.

\section{w Command}

In order to handle the w command we first get a lock on the clients list and, if it is not empty, wait on the threads\_done\_cond condition variable, which is fired on the thread handler when it cleans up a thread and notices that there are no more threads in the list.  Once this happens the thread continues through and allows us to execute additional commands.

\section{EOF Handling}

Although not a command exactly, EOF is handled as requested in the assignment.  When we see an EOF we simply do the same thing as the w command (get a lock on the clients list and wait on a condition if it is not empty) except we exit after we block instead of waiting for new commands.

\end{document}
